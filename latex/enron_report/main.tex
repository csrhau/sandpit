\documentclass[a4paper]{article}
\usepackage{standalone}

\usepackage{pgfplots}
\usepackage{booktabs}
\usepackage{textcomp}    % for \texttextlangle and \texttextrangle macros

\usepgfplotslibrary{dateplot}
\pgfplotsset{compat=newest}

\title{Enron Metrics}
\author{Stephen Roberts}

\begin{document}
\maketitle
\begin{abstract}
  This document describes several basic communication patterns found in the Enron dataset.
  Students should be able to reproduce at least some of these results using their software.
\end{abstract}

\section{Message Rates}
The events of the Enron scandal clearly impacted the rates at which e-mails were sent within the firm.
Students were asked to plot message rates for both the whole firm and individual users. 

\begin{figure}
\centering
\documentclass[tikz]{standalone}
\usepackage{pgfplots}
\usepackage{mathpazo}
\usepgfplotslibrary{dateplot}
\pgfplotsset{compat=newest}
% abbreviated month names as tick labels in PGFPlots
% http://tex.stackexchange.com/questions/20445/abbreviated-month-names-as-tick-labels-in-pgfplots


\begin{document}

\begin{tikzpicture}
  \begin{axis}[xlabel={Date}, 
              date coordinates in=x,
              ylabel={Messages/Week},
              xlabel shift = 5 pt,
              width=0.9\textwidth,
              height=0.6\textwidth,
              enlarge x limits=false,
              ymin=0,
              ymax=600,
              xmin=1999-01-01,
              ytick={0,100,200,300,400,500},
              extra x ticks={1999-06-01, 2000-06-01, 2001-06-01, 2002-03-01},
              extra x tick style={
                   yshift=-3.5ex,
                   xticklabel=\year,
                   xticklabel style={name={}},
                   major tick length=0pt,
                   tick label style={rotate=270, font=\small}
              },
              xticklabel style = {font=\tiny, rotate=90}, 
              xtick={1999-01-01,1999-02-01,1999-...-01,1999-12-01,
                     2000-01-01,2000-02-01,2000-...-01,2000-12-01,
                     2001-01-01,2001-02-01,2001-...-01,2001-12-01,
                     2002-01-01,2002-02-01,2002-...-01,2002-12-01},
             xticklabel={\pgfcalendar{tickcal}{\tick}{\tick}{\pgfcalendarshorthand{m}{.}}}]
    \addplot table [x=date,
                    y=rate, 
                    trim cells=true,
                    mark=none,
                    col sep=comma] {auto/data/rates.csv};

  \end{axis}
\end{tikzpicture}
\end{document}

\caption{Enron Corpus Message Rates}
\label{fig:technique}
\end{figure}



\section{Alias Detection}
Many Enron employees had multiple e-mail addresses, both professional and personal.
There are several approaches to detecting aliases in the data. 
These range from simple fuzzy-matching of e-mail addresses to advanced techniques like authorship identification via text mining or node similarity metrics based on graph structure.
The aliases given in AUTOREF were generated with the former simplistic approach. 


\begin{tabular}{ll}
\toprule
Employee & Aliases \\
\midrule
Jeff Dasovich & jdasovic@enron.com \\
James Derrick Jr. & jr..legal@enron.com \\
Vince J Kaminski & kaminski@enron.com, vkaminski@aol.com \\
John J Lavorato &  lavorato@enron.com \\
Ken Lay & chairman.ken@enron.com, klay@enron.com \\
Albert Meyers & bert.meyers@enron.com \\
Mark E Taylor & legal \textlangle.taylor@enron.com\textrangle \\
Kim S Ward & houston \textlangle.ward@enron.com\textrangle \\
Jason Williams & trading \textlangle.williams@enron.com\textrangle \\
Bill Williams III & bill.iii@enron.com \\
\bottomrule 
\end{tabular}



\end{document}
