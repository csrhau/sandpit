\documentclass[a4paper]{article}
\usepackage{standalone}

\usepackage{pgfplots}
\usepackage{booktabs}
\usepackage{multirow}
\usepackage{hyperref}
\usepackage{textcomp}    % for \texttextlangle and \texttextrangle macros

\usepgfplotslibrary{dateplot}
\pgfplotsset{compat=newest}

\title{Enron Metrics}

\begin{document}
\maketitle
\begin{abstract}
  This document describes several basic communication patterns found in the Enron dataset.
  Students should be able to reproduce at least some of these results using their software.
\end{abstract}

\section{Enron E-Mail Dataset}
The first task in this coursework was to read in and parse the Enron e-mail corpus. 
Students were provided with a basic parsing script, with the expectation that they would adapt it as necessary.

\subsection{Data Quality}
The Enron E-Mail corpus is a real-world dataset complete with data quality issues, including:
\begin{itemize}
  \item \textbf{Duplicates:} identical messages appear multiple times;
  \item \textbf{Encoding:} files are are encoded with the Latin-1 character set;
  \item \textbf{Inconsistent Structure:} there is no standard mailbox structure;
  \item \textbf{Aliases:} employees can have multiple e-mail addresses;
  \item \textbf{Partial Data:} the 'To' field is missing from many messages; and 
  \item \textbf{Incomplete Data:} only a subset of Enron employees are represented.
\end{itemize}

\subsection{E-mail aliases}
\label{sub:dummy}
\begin{table}
  \centering
  \begin{tabular}{ll}
  \toprule
  Employee & Aliases \\
  \midrule
  Jeff Dasovich & jdasovic@enron.com \\
  James Derrick Jr. & jr..legal@enron.com \\
  Vince J Kaminski & kaminski@enron.com, vkaminski@aol.com \\
  John J Lavorato &  lavorato@enron.com \\
  Ken Lay & chairman.ken@enron.com, klay@enron.com \\
  Albert Meyers & bert.meyers@enron.com \\
  Mark E Taylor & legal \textlangle.taylor@enron.com\textrangle \\
  Kim S Ward & houston \textlangle.ward@enron.com\textrangle \\
  Jason Williams & trading \textlangle.williams@enron.com\textrangle \\
  Bill Williams III & bill.iii@enron.com \\
  \bottomrule 
  \end{tabular}
  \caption{E-mail aliases}
  \label{tab:aliases}
\end{table}


One particularly severe data quality issue is the problem of aliases; many e-mail addresses can map to a single employee.
There are several approaches to detecting these aliases in the data. 
Techniques range from simple fuzzy-matching of e-mail addresses to advanced techniques like authorship identification via text mining or node similarity metrics.
The aliases given in \autoref{tab:aliases} were generated with the former, more simplistic approach. 

\section{Message Rate Analysis}
The events of the Enron scandal clearly impacted the rates at which e-mails were sent within the firm.
Students were asked to plot message rates for both the whole firm and individual users. 

\begin{table}
  \centering
  \begin{tabular}{ll}
  \toprule
  Week Starting & Significant Events \\
  \midrule
  Aug 2000 & Enron shares reach high of \$90 \\
  Dec 2000 & Jeff Skilling succeeds Ken Lay as CEO \\
  13 Aug 01 & Skilling Resigns; Lay returns as CEO \\
  22 Aug 01 & Sherron Watkins warns Lay of financial irregularities \\
  15 Oct 01 & Enron announces \$638 millin in Q3 losses \\
  22 Oct 01 & SEC launches an inquiry into Enron's finances \\
  29 Oct 01 & SEC inquiry becomes a formal investigation \\
  05 Nov 01 & Dynergy announces Enron aquisition bid \\
  26 Nov 01 & Dynergy retracts its bid, Enron files for bankruptcy \\
  03 Dec 01 & Enron announces 4,000 redundancies \\
  10 Dec 01 & Enron formally accused of violating securities law \\
  31 Dec 01 & Rumours of a possible DoJ criminal investigation surface \\
  07 Jan 02 & DoJ announces criminal investigation \\
            & Auditors Arthur Andersen admit shredding evidence \\
  14 Jan 02 & D.B. Duncan fired from Arthur Andersen \\
  21 Jan 02 & Former CEO of Enron North America commits suicide \\
  \bottomrule
  \end{tabular}
  \caption{Enron Scandal Timeline}
\end{table}

\begin{figure}
\centering
\documentclass[tikz]{standalone}
\usepackage{pgfplots}
\usepackage{mathpazo}
\usepgfplotslibrary{dateplot}
\pgfplotsset{compat=newest}
% abbreviated month names as tick labels in PGFPlots
% http://tex.stackexchange.com/questions/20445/abbreviated-month-names-as-tick-labels-in-pgfplots


\begin{document}

\begin{tikzpicture}
  \begin{axis}[xlabel={Date}, 
              date coordinates in=x,
              ylabel={Messages/Week},
              xlabel shift = 5 pt,
              width=0.9\textwidth,
              height=0.6\textwidth,
              enlarge x limits=false,
              ymin=0,
              ymax=600,
              xmin=1999-01-01,
              ytick={0,100,200,300,400,500},
              extra x ticks={1999-06-01, 2000-06-01, 2001-06-01, 2002-03-01},
              extra x tick style={
                   yshift=-3.5ex,
                   xticklabel=\year,
                   xticklabel style={name={}},
                   major tick length=0pt,
                   tick label style={rotate=270, font=\small}
              },
              xticklabel style = {font=\tiny, rotate=90}, 
              xtick={1999-01-01,1999-02-01,1999-...-01,1999-12-01,
                     2000-01-01,2000-02-01,2000-...-01,2000-12-01,
                     2001-01-01,2001-02-01,2001-...-01,2001-12-01,
                     2002-01-01,2002-02-01,2002-...-01,2002-12-01},
             xticklabel={\pgfcalendar{tickcal}{\tick}{\tick}{\pgfcalendarshorthand{m}{.}}}]
    \addplot table [x=date,
                    y=rate, 
                    trim cells=true,
                    mark=none,
                    col sep=comma] {auto/data/rates.csv};

  \end{axis}
\end{tikzpicture}
\end{document}

\caption{Enron Corpus Message Rates}
\label{fig:technique}
\end{figure}


\end{document}
